\section{Introduction}
\label{SEC:introduction}

Password database thefts are increasingly common and continue to put users at
risk. Both individuals and businesses want passwords to be stored efficiently,
yet in a way that minimizes the impact of a password database disclosure. Major
corporations, in particular, have encountered significant challenges in trying
to protect their customers' and clients' identities and accounts. Reflecting
the reach of the problem, a hacker on a Russian forum was selling a database
with more than a billion password hashes obtained from different password
databases, primarily through SQL injection \cite{miranteTR13}. Some companies
and organizations that have been successfully hacked include: the Social
Security Administration, Hotmail, LastFM, Formspring, ScribD, the New York
Times, Nvidia, Evernote, LinkedIn, Billabong, Gawker, Linode, ABC, Yahoo!,
eHarmony, LivingSocial, and Twitter~\cite{miranteTR13}. Globally -- despite
ongoing efforts to enhance password security and adhere to best practices --
password database breaches continue. 

Notwithstanding new technologies such as authentication tags, biometric
technologies, and NFC-capable devices, the most prevalent user authentication
method remains the user password. Security experts have long advocated that
user passwords not be stored in plain text but rather, that they be reduced to
salted hashes before storing.  A salted hash consists of a secure hash and a
salt.  The secure hash acts as a one-way function that ensures that an attacker
cannot easily read the plain-text passwords from disk. Salting inserts a random
value that complicates the use of tables that allow hackers to immediately look
up passwords to match most hashes.  Storing passwords with a salted hash is
widely considered to be the best practice because of the level of protection
offered and the ease of implementation (e.g., salted hashing requires no
additional hardware or client software).

However, storing passwords with a salted hash is not a panacea.  That is, when
stored password data has been compromised, attackers have proven themselves adept at
quickly cracking large numbers of passwords, even when salted-password hashes
comprise the stored data. For example, Troy Hunt, a security researcher, showed
that cracking 60\% of a 40,000-entry salted-SHA1 database can be done in a
couple of hours, using a couple of GPU’s and an average computer~\cite{thunt-hashing}. 

This paper presents PolyPasswordHasher, a new technique that makes cracking
individual passwords infeasible because the stored password-entries are 
interdependent.  We leverage cryptographic hashing and threshold cryptography 
to combine 
password hash data with shares so that users unknowingly protect each other's password
data.  With a small amount of additional work by the server, 
PolyPasswordHasher increases, by many orders of magnitude, the time an
attacker needs to crack passwords.  

PolyPasswordHasher is designed to be easy for organizations to adopt.  Much
like salted hashing, PolyPasswordHasher is a software-only, single-server
enhancement that can be deployed on a server without any changes to clients.
PolyPasswordHasher relies on simple primitives that are efficient from a
storage, memory, and computational standpoint.  It integrates with other forms
of authentication including: OAuth, hardware tokens, two factor authentication,
and fingerprint authentication. PolyPasswordHasher is also easy to implement;
two outside developers, whom we had no prior contact with, independently
built implementations of PolyPasswordHasher, in different programming
languages.  Deploying PolyPasswordHasher is a straightforward software
installation on a server.

The contributions of this paper are as follows: 

\begin{itemize}
    \item We describe our design of PolyPasswordHasher, a server-side, 
software-only scheme that protects against an attacker who is able to 
read all persistent storage on a server, including the complete password file.
    \item We demonstrate the \PPH has similar performance, storage, and
memory requirements to deployed systems.
    \item  We describe a mechanism to shield some user accounts, who
may be untrustworthy or choose poor passwords, as long as the passwords 
from a set of \thresholdaccounts are not cracked.
    \item We analyze different practical scenarios and analyze the properties 
and limitations of \PPH in these environments.
\end{itemize}

This paper is organized into eight sections. Sections~\ref{SEC:threat-model}
and~\ref{SEC:background} introduce the threat model and the theoretical
background of password storage mechanisms.  Sections~\ref{SEC:design}
and~\ref{SEC:implementation} provide the design specification and 
implementation of \PPH. 
%Section~\ref{SUBSEC:deployment} provides information about our deployed django
%application and it’s specific design considerations to provide a feel of how an
%actual PolyPasswordHasher implementation works.  
Section~\ref{SEC:evaluation} evaluates \PPH's suitability and resilience
to cracking in a variety of scenarios; it also
provides information about the implementation’s performance for different kinds
of operations. 
We describe the relation to \PPH with other related work in 
Section~\ref{SEC:related-work}.
Finally, Section~\ref{SEC:conclusion} summarizes and anaylzes our findings 
described in the paper.

