\section{Conclusion}
\label{SEC:conclusion}

This work presents \PPH, a technique for protecting user passwords in the 
event of password database disclosure. \PPH is the first technique
that requires only software changes on the server and yet requires
attackers to do asymmetrically more work to crack passwords than
servers need to do to verify them.  As a result, password
cracking becomes infeasible for attackers in many cases.


\PPH is practical to deploy and is effective in practice.  The performance
of \PPH is similar to salted hashing.  The memory and storage costs from
using \PPH are negligible.  So long as good password selection procedures
are followed, \thresholdaccounts increase an attacker's cracking effort by
many orders of magnitude.  We show how configuring \partialverification
in different ways can help to optimize \PPH for different configurations
of services.

We have installed \PPH at our institution and are gaining practical
experience to uncover any usability problems.  Five different parties, 
two of whom had no prior contact with us, authored an implementation
of \PPH.  There are implementations for a variety of languages and 
web frameworks available with an MIT license at:
\showurlx. 


