\section{Conclusion}
\label{sec-conclusion}

This work addresses an important problem in the security community by 
protecting user passwords in the event of password file disclosure.   
Previously an attacker that obtained a salted and hashed password file could 
easily crack passwords one at a time.   Now,
assuming the attacker must have access to a threshold of passwords to 
crack individual passwords.  As a result,
password cracking requires exponentially more work for the attacker and
is infeasible in many cases.

Our implementation of PolyPasswordHasher demonstrates that the storage and
performance properties are similar to systems that are widely used in practice.
PolyPasswordHasher integrates naturally with alternative authentication 
mechanisms, tools, and techniques.
In on-going work, we are deploying PolyPasswordHasher %in production 
%use 
in several domains including a production web service used by thousands
of users and a cloud computing infrastructure.   
%Our future work will involve studying any
%usability concerns that arise from PolyPasswordHasher's adoption.

Our reference implementation is available with an MIT license at:
\showurlx.
%\url{https://polypasswordhasher.poly.edu}.   

\cappos{Future Work: partially rotate share coverage after unlock to 
minimize attack effectiveness after reboot.

Future Work: Use variable quanta shares to incrementally protect more data.}
%\subsection{Future Work}
%
%
%In our on-going work, we are extending upPIR to better mask file sizes, 
%especially for large updates.   We are exploring different algorithms
%that the vendor may use to pack content into a release.
%We are also extending upPIR
%to recover gracefully from mirrors that serve bogus content.   Our efforts
%also include applying upPIR more broadly than software updates.
%
%We also are pursuing an open standard for upPIR communications to ensure
%interoperation between different client, mirror, and vendor tracker 
%implementations.

