\section{Implementation and Limitations}
\label{SEC:implementation}

Our reference implementation for PolyPasswordHasher is available with an MIT
license at \showurlx.  It utilizes a 16 byte salt, with SHA256, to compute password
hashes. The Shamir Secret Sharing routines utilize GF256 as the underlying
field (by encoding each byte as a separate share).  The code base for
PolyPasswordHasher is 120 lines of Python code.  This code handles the
functionality described in Section~\ref{SEC:design}, including support for protector,
bootstrapping, and \thresholdlessaccounts; reading/writing a database to disk;
changing passwords; and detecting \partialverification of incorrect passwords.
PolyPasswordHasher relies on mathematical libraries such as GF256 operations,
Lagrange interpolation, and polynomial math code -- 391 lines of Python and 83 lines
of C code.  Additionally, the implementation uses standard python
libraries for functionalities such as SHA256 and AES.

Several outside developers, unbeknownst to us (until they shared their work
with us) implemented versions of PolyPasswordHasher in Ruby and PHP.  They
created their implementations based on our Python reference implementation and
specification that is available in our GitHub repository.
Table~\ref{TABLE:source-code} describes these outside PPH implementations.
Other external developers have uploaded fixes and improvements to our code when
they wanted it to be compatible with their environment. 

\begin{table}[t]
    \centering
    \renewcommand{\arraystretch}{1.3}

    \begin{tabular}{| c | c | c | }
    \hline
    {\bf Language } & {\bf Lines of Code} & {\bf Author}\\
    \hline
    C & 598 & Local1\\
    \hline
    Python & 391 (+83) & Local2\\
    \hline
    Django & 786 & Local1 + External1\\
    \hline
    Ruby & 437 & External2\\
    \hline
    PHP* & N/A  & External3\\
    \hline
    \end{tabular}
    \caption{Different implementations of \PPH for different Languages and Frameworks, the
    PHP implementation has not been publicly released}
    \label{TABLE:source-code}
\end{table}



\subsection{Handling Alternate Authentication Mechanisms}
\label{SUBSEC:alternate-authentication-mechanisms}

Passwords are not the only mechanism for logging into modern systems. For
example, it is common for users to have a private key to log in over ssh,
biometric-based authentication holds substantial promise, smart cards are often
used to hide user credentials, and single sign-in systems like OAuth and OpenID
are commonplace and provided by many major websites. Any practical protection
mechanism needs to operate in conjunction with such techniques.

Handling non-password logins for shielded user accounts is trivial because
it requires no changes to the system. These non-password logins can be handled
using existing login mechanisms, without any modification. 

For \thresholdaccounts, it is possible to view the authentication process as
decrypting information using a secret that has been stored by a remote party
(as in fact many such systems are implemented). Essentially, we can encrypt the
Shamir Secret Share with the user’s key and check that the resulting share is
correct.  For example, suppose that an administrator has a public/private
keypair that is used to log in (as is common with SSH private key or smart card
based authentications).  Instead of XORing the Shamir Share with the salted
hash of the password, the share can be encrypted with the user’s private key.
When the administrator presents her public key to login, it decrypts the
original share. 

Prior work on deriving a private key from (noisy) data [34] is relevant when
authenticating \thresholdaccounts using other techniques, such as biometrics.
For example, once a key has been derived from biometric data,
PolyPasswordHasher will function in a way that is identical to the private key
authentication scheme discussed in the previous paragraph. 

\subsection{Deployment}
\label{SUBSEC:deployment}

One key issue with deploying PolyPasswordHasher is deciding which accounts should be protector
accounts and which should be shielded.  General guidelines include that
\thresholdaccounts should be assigned to users who choose strong passwords.
Ideally these are users who will log in soon after a reboot, which will
minimize the amount of time needed to bootstrap. Although different servers may
have different use patterns, in our experience, system administrators usually
meet the above requirements.  We examined logs from servers at our institution
and found that system administrators typically login a few minutes after a
system is restarted (likely because they will check the system to make sure it
is operating correctly).  Based on these practices, system administrators tend
to be ideal candidates for \thresholdaccounts.

The threshold setting is another important value for deployment.  In our
experience, a threshold value of 2 - 5 is sufficient, even for systems that
process thousands of password requests.  As we demonstrate in the evaluation
section, even a threshold of 2 increases password strength immensely.

Using PolyPasswordHasher results in very minor changes to existing password
authentication systems. The client tools for password authentication do not
change in any way. In fact, because PolyPasswordHasher only impacts password
storage, it is invisible to clients. For administrators, the only change is
that when an account is created, the administrator can specify whether or not
an account counts toward the threshold.

The file format for password storage is similar to existing systems. In systems
that use a database, the share number and isolated-check bits can simply be
inserted as new columns in the table.  Many servers (such as Linux, Mac, and
BSD) use the /etc/shadow or /etc/master.password colon delimited formats.  As
it is used today, the ‘password’ field contains both the salt and hash but
these different portions of the field are not delineated.  With
PolyPasswordHasher, additional data (such as the share number and
isolated-check bits) can also be added at a known position within the
‘password’ field (e.g., the beginning or end); doing so will lengthen this
variable length, opaque field.


